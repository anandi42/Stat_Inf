\documentclass[]{article}
\usepackage{lmodern}
\usepackage{amssymb,amsmath}
\usepackage{ifxetex,ifluatex}
\usepackage{fixltx2e} % provides \textsubscript
\ifnum 0\ifxetex 1\fi\ifluatex 1\fi=0 % if pdftex
  \usepackage[T1]{fontenc}
  \usepackage[utf8]{inputenc}
\else % if luatex or xelatex
  \ifxetex
    \usepackage{mathspec}
    \usepackage{xltxtra,xunicode}
  \else
    \usepackage{fontspec}
  \fi
  \defaultfontfeatures{Mapping=tex-text,Scale=MatchLowercase}
  \newcommand{\euro}{€}
\fi
% use upquote if available, for straight quotes in verbatim environments
\IfFileExists{upquote.sty}{\usepackage{upquote}}{}
% use microtype if available
\IfFileExists{microtype.sty}{%
\usepackage{microtype}
\UseMicrotypeSet[protrusion]{basicmath} % disable protrusion for tt fonts
}{}
\usepackage[margin=1in]{geometry}
\usepackage{color}
\usepackage{fancyvrb}
\newcommand{\VerbBar}{|}
\newcommand{\VERB}{\Verb[commandchars=\\\{\}]}
\DefineVerbatimEnvironment{Highlighting}{Verbatim}{commandchars=\\\{\}}
% Add ',fontsize=\small' for more characters per line
\usepackage{framed}
\definecolor{shadecolor}{RGB}{248,248,248}
\newenvironment{Shaded}{\begin{snugshade}}{\end{snugshade}}
\newcommand{\KeywordTok}[1]{\textcolor[rgb]{0.13,0.29,0.53}{\textbf{{#1}}}}
\newcommand{\DataTypeTok}[1]{\textcolor[rgb]{0.13,0.29,0.53}{{#1}}}
\newcommand{\DecValTok}[1]{\textcolor[rgb]{0.00,0.00,0.81}{{#1}}}
\newcommand{\BaseNTok}[1]{\textcolor[rgb]{0.00,0.00,0.81}{{#1}}}
\newcommand{\FloatTok}[1]{\textcolor[rgb]{0.00,0.00,0.81}{{#1}}}
\newcommand{\CharTok}[1]{\textcolor[rgb]{0.31,0.60,0.02}{{#1}}}
\newcommand{\StringTok}[1]{\textcolor[rgb]{0.31,0.60,0.02}{{#1}}}
\newcommand{\CommentTok}[1]{\textcolor[rgb]{0.56,0.35,0.01}{\textit{{#1}}}}
\newcommand{\OtherTok}[1]{\textcolor[rgb]{0.56,0.35,0.01}{{#1}}}
\newcommand{\AlertTok}[1]{\textcolor[rgb]{0.94,0.16,0.16}{{#1}}}
\newcommand{\FunctionTok}[1]{\textcolor[rgb]{0.00,0.00,0.00}{{#1}}}
\newcommand{\RegionMarkerTok}[1]{{#1}}
\newcommand{\ErrorTok}[1]{\textbf{{#1}}}
\newcommand{\NormalTok}[1]{{#1}}
\usepackage{graphicx}
\makeatletter
\def\maxwidth{\ifdim\Gin@nat@width>\linewidth\linewidth\else\Gin@nat@width\fi}
\def\maxheight{\ifdim\Gin@nat@height>\textheight\textheight\else\Gin@nat@height\fi}
\makeatother
% Scale images if necessary, so that they will not overflow the page
% margins by default, and it is still possible to overwrite the defaults
% using explicit options in \includegraphics[width, height, ...]{}
\setkeys{Gin}{width=\maxwidth,height=\maxheight,keepaspectratio}
\ifxetex
  \usepackage[setpagesize=false, % page size defined by xetex
              unicode=false, % unicode breaks when used with xetex
              xetex]{hyperref}
\else
  \usepackage[unicode=true]{hyperref}
\fi
\hypersetup{breaklinks=true,
            bookmarks=true,
            pdfauthor={AKN},
            pdftitle={Exploratory Data Analysis of Tooth Growth in Gerbils},
            colorlinks=true,
            citecolor=blue,
            urlcolor=blue,
            linkcolor=magenta,
            pdfborder={0 0 0}}
\urlstyle{same}  % don't use monospace font for urls
\setlength{\parindent}{0pt}
\setlength{\parskip}{6pt plus 2pt minus 1pt}
\setlength{\emergencystretch}{3em}  % prevent overfull lines
\setcounter{secnumdepth}{0}

%%% Use protect on footnotes to avoid problems with footnotes in titles
\let\rmarkdownfootnote\footnote%
\def\footnote{\protect\rmarkdownfootnote}

%%% Change title format to be more compact
\usepackage{titling}

% Create subtitle command for use in maketitle
\newcommand{\subtitle}[1]{
  \posttitle{
    \begin{center}\large#1\end{center}
    }
}

\setlength{\droptitle}{-2em}
  \title{Exploratory Data Analysis of Tooth Growth in Gerbils}
  \pretitle{\vspace{\droptitle}\centering\huge}
  \posttitle{\par}
\subtitle{Hypothesis Testing and Confidence Intervals}
  \author{AKN}
  \preauthor{\centering\large\emph}
  \postauthor{\par}
  \date{}
  \predate{}\postdate{}



\begin{document}

\maketitle


\section{Overview}\label{overview}

In this report, we use \texttt{ToothGrowth}, a dataset in R, which
contains data from an experiment measuring the length of guinea pig
teeth, or \emph{odontoblasts}, at three dose levels of Vitamin C, either
through Orange Juice or with just ascorbic acid. We will perform
exploratory data analysis, followed by hypothesis testing and calculate
confidence intervals. Finally, we will show that the tooth growth does
seem to increase with increasing dose of Vitamin C, but we cannot
definitively conclude this due to low power.

\section{The Data and Exploratory
Analyses}\label{the-data-and-exploratory-analyses}

First, we load the ToothGrowth data. The data contains 60 observations
in 3 dose levels of 2 supplement types. First, we examine how the
response variable, tooth length, varies by dose level and supplement
type. We can do this with a simple boxplot.

\begin{Shaded}
\begin{Highlighting}[]
\NormalTok{data1 <-}\StringTok{ }\NormalTok{ToothGrowth}
\NormalTok{data1$dose <-}\StringTok{ }\KeywordTok{as.factor}\NormalTok{(data1$dose)}
\end{Highlighting}
\end{Shaded}

\begin{figure}[htbp]
\centering
\includegraphics{CourseProj_part2_files/figure-latex/plots-1.pdf}
\caption{Data Summary}
\end{figure}

Visually, it looks like Tooth Length is longer at higher dose levels,
regardless of the supplement type. However, there may be some effect of
supplement type as well. The experimental design of this study
essentially has two levels. So, we can group the data in few ways, only
by supp, or only by dose. We can take a look at the subgroup means and
variances before we get to hypothesis testing.

\begin{table}[ht]
\centering
\begin{tabular}{lrr}
  \hline
Group & Means & Variances \\ 
  \hline
OJ & 20.663 & 43.633 \\ 
  VC & 16.963 & 68.327 \\ 
  0.5 & 10.605 & 10.605 \\ 
  1 & 19.735 & 19.735 \\ 
  2 & 26.100 & 26.100 \\ 
   \hline
\end{tabular}
\end{table}

\section{Confidence Intervals and Hypothesis
Testing}\label{confidence-intervals-and-hypothesis-testing}

Let's look at the supplement type first. First we compare the mean and
variance at each supplement type. Referring to our tables of means and
variances above, we see that the variances are \texttt{43.6} and
\texttt{68.32}. So to be safe, we set \texttt{var.equal = FALSE}, in our
t-test. We know from the description of the dataset that the groups are
not paired.

\begin{Shaded}
\begin{Highlighting}[]
\NormalTok{test1 <-}\StringTok{ }\KeywordTok{t.test}\NormalTok{(len ~}\StringTok{ }\NormalTok{supp,}\DataTypeTok{data=}\NormalTok{data1, }\DataTypeTok{var.equal =} \OtherTok{FALSE}\NormalTok{, }\DataTypeTok{paired =} \OtherTok{FALSE}\NormalTok{)}
\NormalTok{test1$conf.int}
\NormalTok{test1$p.value}
\NormalTok{test1$estimate}
\end{Highlighting}
\end{Shaded}

\begin{verbatim}
## [1] -0.1710156  7.5710156
## attr(,"conf.level")
## [1] 0.95
## [1] 0.06063451
## mean in group OJ mean in group VC 
##         20.66333         16.96333
\end{verbatim}

Although the p-value is \texttt{0.06}, the confidence interval contains
\texttt{0}. Because the 95\% CI includes 0, we cannot reject the null
hypothesis that the difference in means is 0. Next, we will test the
group differences by dose level, using \texttt{t.test} as before. Again,
to be safe, based on the calculated group variances, we set
\texttt{var.equal=FALSE}.

\begin{Shaded}
\begin{Highlighting}[]
\NormalTok{dlong <-}\StringTok{ }\KeywordTok{melt}\NormalTok{(data1, }\DataTypeTok{id.vars =} \KeywordTok{c}\NormalTok{(}\StringTok{"dose"}\NormalTok{), }\DataTypeTok{measure.vars=}\StringTok{"len"}\NormalTok{)}
\NormalTok{g1 <-}\StringTok{ }\NormalTok{dlong[dlong$dose ==}\StringTok{ "0.5"}\NormalTok{,]$value}
\NormalTok{g2 <-}\StringTok{ }\NormalTok{dlong[dlong$dose ==}\StringTok{ "1"}\NormalTok{,]$value}
\NormalTok{g3 <-}\StringTok{ }\NormalTok{dlong[dlong$dose ==}\StringTok{ "2"}\NormalTok{,]$value}
\NormalTok{t1 <-}\StringTok{ }\KeywordTok{t.test}\NormalTok{(g2, g1,}\DataTypeTok{var.equal =} \OtherTok{FALSE}\NormalTok{, }\DataTypeTok{paired =} \OtherTok{FALSE}\NormalTok{, }\DataTypeTok{alternative =} \StringTok{"two.sided"}\NormalTok{)}
\NormalTok{t2 <-}\StringTok{ }\KeywordTok{t.test}\NormalTok{(g3, g1,}\DataTypeTok{var.equal =} \OtherTok{FALSE}\NormalTok{, }\DataTypeTok{paired =} \OtherTok{FALSE}\NormalTok{, }\DataTypeTok{alternative =} \StringTok{"two.sided"}\NormalTok{)}
\NormalTok{t3 <-}\StringTok{ }\KeywordTok{t.test}\NormalTok{(g3, g2,}\DataTypeTok{var.equal =} \OtherTok{FALSE}\NormalTok{, }\DataTypeTok{paired =} \OtherTok{FALSE}\NormalTok{, }\DataTypeTok{alternative =} \StringTok{"two.sided"}\NormalTok{)}
\NormalTok{conf_int <-}\StringTok{ }\KeywordTok{rbind}\NormalTok{(t1$conf, t2$conf, t3$conf)}
\NormalTok{p_vals <-}\StringTok{ }\KeywordTok{format.pval}\NormalTok{(}\KeywordTok{c}\NormalTok{(t1$p.value, t2$p.value, t3$p.value), }\DataTypeTok{eps =} \NormalTok{.}\DecValTok{001}\NormalTok{, }\DataTypeTok{digits=}\DecValTok{3}\NormalTok{)}
\NormalTok{t_stat <-}\StringTok{ }\KeywordTok{rbind}\NormalTok{(t1$statistic, t2$statistic,t3$statistic)}
\NormalTok{labels <-}\KeywordTok{c}\NormalTok{(}\StringTok{"1mg vs. 0.5mg"}\NormalTok{, }\StringTok{"2mg vs. 0.5mg"}\NormalTok{, }\StringTok{"2mg vs 1mg"}\NormalTok{)}
\NormalTok{results1 <-}\StringTok{ }\KeywordTok{data.frame}\NormalTok{(labels, conf_int, p_vals, t_stat)}
\KeywordTok{colnames}\NormalTok{(results1) <-}\StringTok{ }\KeywordTok{c}\NormalTok{(}\StringTok{"Comparison"}\NormalTok{, }\StringTok{"95% CI"}\NormalTok{, }\StringTok{"95% CI"}\NormalTok{, }\StringTok{"p"}\NormalTok{, }\StringTok{"t"}\NormalTok{)}
\KeywordTok{print}\NormalTok{(}\KeywordTok{xtable}\NormalTok{(results1, }\DataTypeTok{digits =} \DecValTok{3}\NormalTok{), }\DataTypeTok{comment=}\OtherTok{FALSE}\NormalTok{,}\DataTypeTok{include.rownames=}\OtherTok{FALSE}\NormalTok{)}
\end{Highlighting}
\end{Shaded}

\begin{table}[ht]
\centering
\begin{tabular}{lrrlr}
  \hline
Comparison & 95\% CI & 95\% CI & p & t \\ 
  \hline
1mg vs. 0.5mg & 6.276 & 11.984 & $<$0.001 & 6.477 \\ 
  2mg vs. 0.5mg & 12.834 & 18.156 & $<$0.001 & 11.799 \\ 
  2mg vs 1mg & 3.734 & 8.996 & $<$0.001 & 4.900 \\ 
   \hline
\end{tabular}
\end{table}

We have now done 4 successive t-tests on the data, comparing various
combinations of groups. But what about power? We have estimates of the
`true' mean difference and its variation from our data.

\begin{Shaded}
\begin{Highlighting}[]
\NormalTok{meandiff <-}\StringTok{ }\KeywordTok{c}\NormalTok{(}\KeywordTok{mean}\NormalTok{(g3)-}\KeywordTok{mean}\NormalTok{(g2), }\KeywordTok{mean}\NormalTok{(g3)-}\KeywordTok{mean}\NormalTok{(g1), }\KeywordTok{mean}\NormalTok{(g2-g1))}
\NormalTok{vars1 <-}\StringTok{ }\KeywordTok{c}\NormalTok{(}\KeywordTok{var}\NormalTok{(g1), }\KeywordTok{var}\NormalTok{(g2), }\KeywordTok{var}\NormalTok{(g3))}
\NormalTok{sd1 <-}\StringTok{ }\KeywordTok{sqrt}\NormalTok{(vars1)}
\KeywordTok{rbind}\NormalTok{(meandiff, sd1)}
\end{Highlighting}
\end{Shaded}

\begin{verbatim}
##              [,1]      [,2]    [,3]
## meandiff 6.365000 15.495000 9.13000
## sd1      4.499763  4.415436 3.77415
\end{verbatim}

\begin{Shaded}
\begin{Highlighting}[]
\KeywordTok{power.t.test}\NormalTok{(}\DataTypeTok{n=}\DecValTok{10}\NormalTok{, }\DataTypeTok{type =} \StringTok{"two.sample"}\NormalTok{, }\DataTypeTok{alternative =} \StringTok{"two.sided"}\NormalTok{, }
             \DataTypeTok{sig.level =} \FloatTok{0.01}\NormalTok{, }\DataTypeTok{sd=}\FloatTok{4.5}\NormalTok{, }\DataTypeTok{delta=}\FloatTok{6.4}\NormalTok{)$power}
\end{Highlighting}
\end{Shaded}

\begin{verbatim}
## [1] 0.6218381
\end{verbatim}

\begin{Shaded}
\begin{Highlighting}[]
\KeywordTok{power.t.test}\NormalTok{(}\DataTypeTok{power =} \FloatTok{0.95}\NormalTok{, }\DataTypeTok{type =} \StringTok{"two.sample"}\NormalTok{, }\DataTypeTok{alternative =} \StringTok{"two.sided"}\NormalTok{, }
             \DataTypeTok{sig.level =} \FloatTok{0.01}\NormalTok{, }\DataTypeTok{sd=}\FloatTok{4.5}\NormalTok{, }\DataTypeTok{delta=}\FloatTok{6.4}\NormalTok{)$n}
\end{Highlighting}
\end{Shaded}

\begin{verbatim}
## [1] 19.35302
\end{verbatim}

At an \(n\) of 10, and \(\alpha\) of 0.01 we have 62\% power to
correctly reject a null hypothesis assuming a true mean difference of
\textasciitilde{}6.4 (smallest diff observed). To achieve power of 95\%,
at least 19 gerbils per group are needed.

\section{Conclusions}\label{conclusions}

\begin{enumerate}
\def\labelenumi{\arabic{enumi}.}
\itemsep1pt\parskip0pt\parsep0pt
\item
  When comparing the means at each dose level (0.5, 1 and 2 mg), all 3
  combinations of comaprisons produce confidence intervals that do not
  contain 0, t \textgreater{} 1, and p \textless{} 0.001.
\item
  We infer from this data that while the \emph{type} of Vitamin C
  supplement \emph{does not} significantly affect tooth growth, higher
  \emph{doses} of Vitamin C \emph{may have an effect on tooth growth}.
\item
  We may be underpowered, as we are at \textless{}75\% power to
  correctly reject a null hypothesis when the true mean difference is
  equal to the smallest difference we observed.
\end{enumerate}

\section{Assumptions}\label{assumptions}

\begin{enumerate}
\def\labelenumi{\arabic{enumi}.}
\itemsep1pt\parskip0pt\parsep0pt
\item
  We assumed that the variances across the group (either by dose or
  supplement) was not equal.
\item
  We assumed that the 60 gerbils used for this data were randomy
  assigned to each of the dose/supplement groups.
\item
  We assumed that the 60 gerbils are representative of the entire
  population of gerbils.
\item
  Finally, we assumed that the response variable is not itself affected
  by other variables that we have not included in our statistical
  inference.
\end{enumerate}

\end{document}
